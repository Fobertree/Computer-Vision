\documentclass[10pt, oneside]{article} 
\usepackage{amsmath, amsthm, amssymb, calrsfs, wasysym, verbatim, bbm, color, graphics, geometry, color, mathtools}

\geometry{tmargin=.75in, bmargin=.75in, lmargin=.75in, rmargin = .75in}  

\newcommand{\R}{\mathbb{R}}
\newcommand{\C}{\mathbb{C}}
\newcommand{\Z}{\mathbb{Z}}
\newcommand{\N}{\mathbb{N}}
\newcommand{\Q}{\mathbb{Q}}
\newcommand{\F}{\mathbb{F}}
\newcommand{\I}{\mathcal{I}}
\newcommand{\Cdot}{\boldsymbol{\cdot}}
\newcommand{\inc}{\Delta}
\newcommand\norm[1]{\lVert#1\rVert}
\newcommand\normx[1]{\Vert#1\Vert}

% fix for mathptmx screwing up mathcal
\DeclareMathAlphabet{\mathcal}{OMS}{cmsy}{m}{n}

\newtheorem{thm}{Theorem}
\newtheorem{defn}{Definition}
\newtheorem{conv}{Convention}
\newtheorem{rem}{Remark}
\newtheorem{lem}{Lemma}
\newtheorem{cor}{Corollary}
\newcommand{\highlight}[1]{%
\colorbox{red!50}{$\displaystyle#1$}}

\usepackage[hidelinks]{hyperref}

\title{CS 485 - Computer Vision}
\author{Alexander Liu}
\date{Fall 2025}

\begin{document}

\maketitle
\tableofcontents

\vspace{.25in}
\newpage

\section{Introduction}
\subsection{Example Application}
Telepresence (Google Project Starline)
\begin{itemize}
    \item Enable interaction as if in the same space
    \item Stereo camera - two slightly different angles (stereo pairs)
\end{itemize}
Nano Banana
\begin{itemize}
    \item Modify image with text
\end{itemize}
Novel view synthesis (Adobe Project Turntable)
\begin{itemize}
    \item Create entirely new asset (manipulate 2D image)
\end{itemize}
3D World Reconstruction (Nvidia GEN3C)
\begin{itemize}
    \item Point cloud to video
\end{itemize}
What is Computer Vision?
\begin{itemize}
    \item Enabling machines to perceive world around us
\end{itemize}
Final Project
\begin{itemize}
    \item Conventional
    \begin{itemize}
        \item Structure-from-Motion (SFM)
    \end{itemize}
    \item ML-based
    \begin{itemize}
        \item NeRF / 3D Gaussians
        \begin{itemize}
            \item 3D Reconstruction of favorite places
            \item 4D reconstruction from yuor own captured videos
        \end{itemize}
    \end{itemize}
\end{itemize}

\section{Images and Filters}
Shorter wavelengths: violet. Longer wavelengths: Red

Photons
\begin{itemize}
    \item Absorption
    \item Diffusion
    \begin{itemize}
        \item Randomized reflection vectors
    \end{itemize}
    \item Specular reflation
    \begin{itemize}
        \item Single ray
    \end{itemize}
    \item Transparency
    \item Refraction
    \item Fluorescence
    \item Subsurface scattering
    \begin{itemize}
        \item Diffuse below the surface (can see effect as light going through human skin)
    \end{itemize}
    \item Phosphorescence
    \item Interreflection
\end{itemize}
BRDF (bidirectional reflectance distribution function)
\begin{itemize}
    \item Describes how light reflects off a surface
\end{itemize}
Rods
\begin{itemize}
    \item rod-shapes
    \item more sensitive
    \item gray-scale
\end{itemize}
Cones
\begin{itemize}
    \item Detects color
    \item Three types
    \begin{itemize}
        \item S-: blue
        \item M-: green/yellow
        \item L-: red light
    \end{itemize}
\end{itemize}
Three types of images
\begin{itemize}
    \item Binary: 0/1
    \item Grayscale: 0-255
    \item Color: RGB
\end{itemize}
Resolution
\begin{itemize}
    \item sampling parameter, in dots per inch (DPI)
\end{itemize}
Images are Sampled and Quantized
\begin{itemize}
    \item Quantized: constrain values to range (in this case [0,255])
\end{itemize}
\subsection{Onto Core Content: Filters}
Filtering
\begin{itemize}
    \item Forming a new image whose pixel values are transformed from original pixel values
    \item Ex:
    \begin{itemize}
        \item De-noising
        \item Super-resolution
        \item In-painting
    \end{itemize}
\end{itemize}
\subsubsection{2D Convolution}
Filtering operation between kernel $f$ and $h$
\begin{itemize}
    \item Flipping kernel $h$ (both horizontally and vertically), then slide it over image $f$, multiplying overlapping elements, and summing
    \begin{itemize}
        \item Why does it flip?
        \item Something to do with associativity
    \end{itemize}
    \item Steps
    \begin{itemize}
        \item Flip h vertically. Flip h horizontally
        \item Shift flipped results by $n,m$ to form $h[n-k,m-l]$
    \end{itemize}
    \item Note: zero-padding to preserve dimensions, otherwise decrease dims by 2 or stride
\end{itemize}

\section{Filters and Edges}
Recall convolution:
\[f[n,m]*h[n,m] = \sum_k\sum_l f[k,l]\cdot h[n-k,m-l]\]
Flip kernel first. Why?
\begin{itemize}
    \item Associativity
    \item Following definition of convolution. Otherwise, it is cross-correlation
\end{itemize}
\subsection{Correlation}
Cross correlation is convolution but WITHOUT the flip
\begin{itemize}
    \item Slide template over sample image with noisy filter applied. Use numpy's correlate to find specific values
    \item Applying an initial filter to keep values above certain level, does a better job at distinguishing values
\end{itemize}
Why correlation peaks at a match?
\begin{itemize}
    \item Correlation = multiply template with image patch, then sum
    \item Mismatch $\to$ products cancel out $\to$ small score
    \item Good match $\to$ products reinforce, high sum
    \item Highest sum when template aligns with image (similarity score)
\end{itemize}
Convolution is a filtering operation, correlation is a measure of relatedness of two signals\vspace{0.15in}\\
\subsection{Edge detection}
Origin of edges (4 main sources)
\begin{itemize}
    \item Surface normal discontuinity
    \begin{itemize}
        \item surface normal-facing direction changes
    \end{itemize}
    \item Depth discontuinity
    \begin{itemize}
        \item Foreground vs background (depth)
    \end{itemize}
    \item Surface color discontinuity
    \begin{itemize}
        \item Diff. color
    \end{itemize}
    \item Illumination discontinuity
    \begin{itemize}
        \item Lights
    \end{itemize}
\end{itemize}
Finite differences in 1D (derivative)
\begin{itemize}
    \item Backward
    \item Forward
    \item Central
\end{itemize}
We can use these finite differences to create filters
\begin{itemize}
    \item Backward filter: $f(x)-f(x-1)=f^\prime(x)$
    \begin{itemize}
        \item $[0,1,-1]$
    \end{itemize}
    \item Forward filter: $f(x+1)-f(x)=f^\prime(x)$
    \begin{itemize}
        \item $[1,-1,0]$
    \end{itemize}
    \item Central filter: $f(x+1)-f(x-1) = f^\prime(x)$
    \begin{itemize}
        \item $[1,0,-1]$
    \end{itemize}
    \item Leading positive term is assigned $+1$, negative term is assigned $-1$, what isn't being used is assigned $0$
    \item $[sign(x-1), sign(x), sign(x+1)]$, then flip kernel by convolution definition
\end{itemize}
Discrete derivatives in 2D
\begin{itemize}
    \item Gradient
    \begin{itemize}
        \item $\nabla f(x,y) = [f_x, f_y]^T$
        \item gradient magnitude: $\left | \nabla f(x,y)\right | = \sqrt{f_x^2+f_y^2}$
        \item Gradient direction: $\theta = \tan^{-1}\left(\frac{f_y}{f_x}\right)$
    \end{itemize}
\end{itemize}
What does this filter do?
\[\frac{1}{3}\begin{bmatrix}
    1&0&-1\\
    1&0&-1\\
    1&0&-1
\end{bmatrix}\]
\begin{itemize}
    \item x-derivative filter, highlights vertical edges
    \item Transposed version highlights horizontal edges
\end{itemize}
Simple edge detector
An edge is a place of rapid change in the image intensity function\vspace{0.15in}\\
Gradient of an image: $\nabla f = [f_x, f_y]$
\begin{itemize}
    \item Recall: gradient vector points in direction of greatest increase
    \item Edge strength is given by the gradient magnitude
    \begin{itemize}
        \item $\|\nabla f\| = \sqrt{f_x^2+f_y^2}$
    \end{itemize}
\end{itemize}
Effects of noise
\begin{itemize}
    \item Problem: discrete gradient filters respond strongly to noise
    \item What can be done?
    \begin{itemize}
        \item Smooth the image before taking the image, apply smoothing filter to make pixels look more like neighbors
    \end{itemize}
    \item Smoothing filters
    \begin{itemize}
        \item Mean smoothing
        \item Gaussian smoothing
    \end{itemize}
\end{itemize}
Derivative theorem of convolution
\[\frac{d}{dx}(f * g) = f * \frac{d}{dx}g\]
\begin{itemize}
    \item Can pre-compute/take derivative of Gaussian filter
\end{itemize}
Tradeoff between smoothing and localization
\begin{itemize}
    \item Stronger smoothing removes noise, but blurs edges
    \begin{itemize}
        \item Increase kernel size, removes noise, but blurs edges and vice-versa
    \end{itemize}
\end{itemize}
Designing an edge detector
\begin{itemize}
    \item Criteria for an "optimal" edge detector
    \begin{itemize}
        \item Good detection: minimize probability of false positives (i.e., detecting spurious noise as an edge)
        \item Good localization: detected edges must be as close as possible to actual edge
        \item Single response: one point for each true edge point (one-to-one)
    \end{itemize}
\end{itemize}
\subsection{Sobel Edge Detector}
\begin{itemize}
    \item Uses two 3x3 kernels (one horizontal, one vertical) which are convolved with original image
    \[G_x = \begin{bmatrix}
        +1&0&-1\\
        +2&0&-2\\
        +1&0&-1
    \end{bmatrix}\]
    \[G_y = \begin{bmatrix}
        +1&+2&-1\\
        0&0&0\\
        -1&-2&-1
    \end{bmatrix}\]
\end{itemize}
Magnitude:
\[\sqrt{G_x^2 + G_y^2}\]
Sobel Filter Problems
\begin{itemize}
    \item Poor localization (can trigger response in multiple adjacent pixels (thick edges)
    \item Thresholding value favors certain directions over others
    \begin{itemize}
        \item Can miss diagonal directions
        \item This means risk of false negatives
    \end{itemize}
\end{itemize}
Sobel filters are derived form the derivative
\[G_x = \begin{bmatrix}
    +1&0&-1\\
    +1&0&-1\\
    +1&0&-1\\
\end{bmatrix} = \begin{bmatrix}
    1\\2\\1
\end{bmatrix}\begin{bmatrix}
    +1&0&1
\end{bmatrix}\]
\begin{itemize}
    \item First vector: Gaussian smoothing
    \item Second vector: derivative
\end{itemize}

\subsection{Canny edge detector}
Steps
\begin{itemize}
    \item Suppress Noise
    \item Compute gradient magnitude and direction
    \item Apply Non-Maximum Suppression
    \item Use hysteresis and connectivity analysis to detect edges
\end{itemize}
Pre-compute derivative of Gaussian filter 
\begin{itemize}
    \item These turn out to be Sobel filters
    \item $\theta = \tan^{-1}\left(\frac{f_y}{f_x}\right)$
\end{itemize}
To make detected edges thinner, apply non-maximum suppression
\begin{itemize}
    \item Key principle: edge should occur where gradient reaches a maxima
    \item Compare current pixel vs neighbors along direction of gradient
    \begin{itemize}
        \item Remove if not maximum (even if it passes threshold)
    \end{itemize}
\end{itemize}
Non-maximum suppression formalized:
\[M(x,y) = \begin{cases}
    |\nabla G|(x,y) \text{ if } |\nabla G|(x,y) > |\nabla G|(x^\prime,y^\prime) \& |\nabla G|(x,y) > |\nabla G|(x^{\prime\prime},y^{\prime\prime})\\
    0 \text{ otherwise}
\end{cases}\]
\begin{itemize}
    \item Where $|\nabla G|(x,y)$ is the magnitude at point $(x,y)$ and $(x,y), (x^\prime,y^\prime), (x^{\prime\prime}, y^{\prime\prime})$ lie along gradient direction (normal to edge)
\end{itemize}
Detecting edges with a single threshold
\begin{itemize}
    \item Flawed, use two thresholds
\end{itemize}
Hysteresis thresholding
\begin{itemize}
    \item Avoid streaking near threshold value
    \item Define two thresholds: low and high
    \item If less than low, not an edge
    \item If greater than high, a strong edge
    \item If below low and high, weak edge
    \begin{itemize}
        \item Consider adjacent pixels and check if they are edge pixels. If there is one, declare it an edge pixel
    \end{itemize}
\end{itemize}
Effect of $\sigma$ (Gaussian kernel spread/size)
\begin{itemize}
    \item Can affect coarseness of edges
    \item Large $\sigma$: detect large scale edges
    \item Small $\sigma$: detect fine features
\end{itemize}

\section{Line Detection}
What makes this different from edges?
\begin{itemize}
    \item Edge only give parts where pixels change rapidly, but lines give us structure
\end{itemize}
Fitting as Search in Parametric Space
\begin{itemize}
    \item Parametric model for a set of features
    \item Computational complexity matters
\end{itemize}
Difficulty of line fitting
\begin{itemize}
    \item Edge point clutter
    \item Only some parts of line detected, some fitting
    \item Noise in measured edge points
    \begin{itemize}
        \item How to detect true underlying parameters?
    \end{itemize}
\end{itemize}
\subsection{RANSAC}
Overview
\begin{itemize}
    \item RANdom SAmple Consensus
    \item Approach: Avoid impact of outliers
    \item Iteratively maximize number of inlines
    \begin{itemize}
        \item Least-squares
    \end{itemize}
\end{itemize}
Task: Estimate the best line\\
RANSAC stage (k iterations)
\begin{itemize}
    \item Randomly pick 2 points, create a line from the two points
    \item Find total number of points within a (distance) threshold of line (inlier points)
    \item Repeat, until we get a good result
\end{itemize}
Refinement stage (repeat until the inliner set is stable)
\begin{itemize}
    \item Take all inliners of the best line
    \item Fit a new line using least squares (linear regression)
    \item Re-check which points are inliners
    \item Keep doing this until inliner set is stable (number of inliner points doesn't change, i.e. size of set remains the same)
\end{itemize}
Pros
\begin{itemize}
    \item General method suited for a wide range of model fitting problesm
    \item Easy to implement
\end{itemize}
Cons
\begin{itemize}
    \item Only handles a moderate number of outliers before cost blowing up
    \item Many problems have high rates of outliers
\end{itemize}

\subsection{Hough Transform}
Hough transform can detect lines, circles, and structures ONLY if their parametric equations are known\vspace{0.15in}\\
Naive line detection
\begin{itemize}
    \item For every pair of edge pixels
    \begin{itemize}
        \item Compute equation of line
        \item Check if other pixels satisfy equation
        \item $O(N^2)$ for $N$ edge pixels. Too expensive
    \end{itemize}
\end{itemize}
Idea
\begin{itemize}
    \item Take one edge point $(x_i,y_i)$
    \begin{itemize}
        \item Many potential lines passing through point
        \item Family of lines has form $y_i = a*x_i +b$
    \end{itemize}
    \item Original $(x,y)$ space is in image space
    \[y=a_i*x+b_i\]
    \item $a,b$ values get mapped to ``parameter space" $(a,b)$
    \begin{itemize}
        \item Since $a,b$ are inversely proportional, $(a_1,b_1)$ and $(a_2,b_2)$ produce negative sloping line: $b=-x_i * a + y_i$
        \[b=-x_i*a+y_i\]
        \begin{itemize}
            \item \textbf{Proof for this:} subtract $a_i * x$ from both sides
        \end{itemize}
        \item All the feasible combinations of $(a,b)$ are on the same line in the parameter space
        \item Note that since $x_i$ and $y_i$ are our new fixed constants (our $m$ and $b$ in $y=mx+b$) each point $(x_i,y_i)$ produces a new line in parameter space
    \end{itemize}
\end{itemize}
Detecting lines using Hough transform
\begin{itemize}
    \item Each point produces a new line in the parameter space
    \item The intersection of the lines in the parameter space provide $(a^\prime,b^\prime)$, which forms the line $y=a^\prime*x + b^\prime$, which represents the equation in the original image space that connects the two points
    \begin{itemize}
        \item Proof also by algebra, in moving $ax$ term
    \end{itemize}
    \item Convert parameter space into discretized cells in a grid
    \begin{itemize}
        \item We vote on the discrete cells that are `activated' by the transformed line in $(a^\prime, b^\prime)$. Increment cells it is passing through by $1$
        \item Find cells with more votes than threshold
        \begin{itemize}
            \item Intuitively, finding lines that correspond to the most edge points
        \end{itemize}
    \end{itemize}
    \item Complexity
    \begin{itemize}
        \item Linear on number of edge points $N$
        \item Linear on number of accumulator cells $M$
        \item $O(N\times M)$
    \end{itemize}
\end{itemize}

%%%FOOTER

\section{Addendums}

\end{document}