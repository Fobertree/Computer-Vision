What makes this different from edges?
\begin{itemize}
    \item Edge only give parts where pixels change rapidly, but lines give us structure
\end{itemize}
Fitting as Search in Parametric Space
\begin{itemize}
    \item Parametric model for a set of features
    \item Computational complexity matters
\end{itemize}
Difficulty of line fitting
\begin{itemize}
    \item Edge point clutter
    \item Only some parts of line detected, some fitting
    \item Noise in measured edge points
    \begin{itemize}
        \item How to detect true underlying parameters?
    \end{itemize}
\end{itemize}
\subsection{RANSAC}
Overview
\begin{itemize}
    \item RANdom SAmple Consensus
    \item Approach: Avoid impact of outliers
    \item Iteratively maximize number of inlines
    \begin{itemize}
        \item Least-squares
    \end{itemize}
\end{itemize}
Task: Estimate the best line\\
RANSAC stage (k iterations)
\begin{itemize}
    \item Randomly pick 2 points, create a line from the two points
    \item Find total number of points within a (distance) threshold of line (inlier points)
    \item Repeat, until we get a good result
\end{itemize}
Refinement stage (repeat until the inliner set is stable)
\begin{itemize}
    \item Take all inliners of the best line
    \item Fit a new line using least squares (linear regression)
    \item Re-check which points are inliners
    \item Keep doing this until inliner set is stable (number of inliner points doesn't change, i.e. size of set remains the same)
\end{itemize}
Pros
\begin{itemize}
    \item General method suited for a wide range of model fitting problesm
    \item Easy to implement
\end{itemize}
Cons
\begin{itemize}
    \item Only handles a moderate number of outliers before cost blowing up
    \item Many problems have high rates of outliers
\end{itemize}

\subsection{Hough Transform}
Hough transform can detect lines, circles, and structures ONLY if their parametric equations are known\vspace{0.15in}\\
Naive line detection
\begin{itemize}
    \item For every pair of edge pixels
    \begin{itemize}
        \item Compute equation of line
        \item Check if other pixels satisfy equation
        \item $O(N^2)$ for $N$ edge pixels. Too expensive
    \end{itemize}
\end{itemize}
Idea
\begin{itemize}
    \item Take one edge point $(x_i,y_i)$
    \begin{itemize}
        \item Many potential lines passing through point
        \item Family of lines has form $y_i = a*x_i +b$
    \end{itemize}
    \item Original $(x,y)$ space is in image space
    \[y=a_i*x+b_i\]
    \item $a,b$ values get mapped to ``parameter space" $(a,b)$
    \begin{itemize}
        \item Since $a,b$ are inversely proportional, $(a_1,b_1)$ and $(a_2,b_2)$ produce negative sloping line: $b=-x_i * a + y_i$
        \[b=-x_i*a+y_i\]
        \begin{itemize}
            \item \textbf{Proof for this:} subtract $a_i * x$ from both sides
        \end{itemize}
        \item All the feasible combinations of $(a,b)$ are on the same line in the parameter space
        \item Note that since $x_i$ and $y_i$ are our new fixed constants (our $m$ and $b$ in $y=mx+b$) each point $(x_i,y_i)$ produces a new line in parameter space
    \end{itemize}
\end{itemize}
Detecting lines using Hough transform
\begin{itemize}
    \item Each point produces a new line in the parameter space
    \item The intersection of the lines in the parameter space provide $(a^\prime,b^\prime)$, which forms the line $y=a^\prime*x + b^\prime$, which represents the equation in the original image space that connects the two points
    \begin{itemize}
        \item Proof also by algebra, in moving $ax$ term
    \end{itemize}
    \item Convert parameter space into discretized cells in a grid
    \begin{itemize}
        \item We vote on the discrete cells that are `activated' by the transformed line in $(a^\prime, b^\prime)$. Increment cells it is passing through by $1$
        \item Find cells with more votes than threshold
        \begin{itemize}
            \item Intuitively, finding lines that correspond to the most edge points
        \end{itemize}
    \end{itemize}
    \item Complexity
    \begin{itemize}
        \item Linear on number of edge points $N$
        \item Linear on number of accumulator cells $M$
        \item $O(N\times M)$
    \end{itemize}
\end{itemize}