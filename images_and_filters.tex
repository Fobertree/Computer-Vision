Shorter wavelengths: violet. Longer wavelengths: Red

Photons
\begin{itemize}
    \item Absorption
    \item Diffusion
    \begin{itemize}
        \item Randomized reflection vectors
    \end{itemize}
    \item Specular reflation
    \begin{itemize}
        \item Single ray
    \end{itemize}
    \item Transparency
    \item Refraction
    \item Fluorescence
    \item Subsurface scattering
    \begin{itemize}
        \item Diffuse below the surface (can see effect as light going through human skin)
    \end{itemize}
    \item Phosphorescence
    \item Interreflection
\end{itemize}
BRDF (bidirectional reflectance distribution function)
\begin{itemize}
    \item Describes how light reflects off a surface
\end{itemize}
Rods
\begin{itemize}
    \item rod-shapes
    \item more sensitive
    \item gray-scale
\end{itemize}
Cones
\begin{itemize}
    \item Detects color
    \item Three types
    \begin{itemize}
        \item S-: blue
        \item M-: green/yellow
        \item L-: red light
    \end{itemize}
\end{itemize}
Three types of images
\begin{itemize}
    \item Binary: 0/1
    \item Grayscale: 0-255
    \item Color: RGB
\end{itemize}
Resolution
\begin{itemize}
    \item sampling parameter, in dots per inch (DPI)
\end{itemize}
Images are Sampled and Quantized
\begin{itemize}
    \item Quantized: constrain values to range (in this case [0,255])
\end{itemize}
\subsection{Onto Core Content: Filters}
Filtering
\begin{itemize}
    \item Forming a new image whose pixel values are transformed from original pixel values
    \item Ex:
    \begin{itemize}
        \item De-noising
        \item Super-resolution
        \item In-painting
    \end{itemize}
\end{itemize}
\subsubsection{2D Convolution}
Filtering operation between kernel $f$ and $h$
\begin{itemize}
    \item Flipping kernel $h$ (both horizontally and vertically), then slide it over image $f$, multiplying overlapping elements, and summing
    \begin{itemize}
        \item Why does it flip?
        \item Something to do with associativity
    \end{itemize}
    \item Steps
    \begin{itemize}
        \item Flip h vertically. Flip h horizontally
        \item Shift flipped results by $n,m$ to form $h[n-k,m-l]$
    \end{itemize}
    \item Note: zero-padding to preserve dimensions, otherwise decrease dims by 2 or stride
\end{itemize}